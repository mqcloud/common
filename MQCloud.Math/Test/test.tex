\documentclass{article}
\usepackage{tikz}
\usepackage{dot2texi}
\usetikzlibrary{calc,arrows,shapes,automata,petri,positioning, decorations.markings, decorations.pathreplacing}  
\tikzset{
    place/.style={
        circle,
        thick,
        minimum size=6mm,
        draw
    },
    transitionV/.style={
        rectangle,
        thick,
        fill=black,
        minimum height=6mm,
        inner xsep=1pt
    }
}

\begin{document}
\pgfdeclarelayer{background}
\pgfdeclarelayer{foreground}
\pgfsetlayers{background,main,foreground}

\begin{tikzpicture}[>=latex',scale=0.8]

    \tikzstyle{n} = [draw=black!95,very thick,shape=circle,minimum size=2em,
                        inner sep=0pt,fill=white!20]
    \begin{dot2tex}[dot,tikz,codeonly,styleonly,options=-s -tmath ]
        digraph G  {
 d2tdocpreamble = "\usetikzlibrary{automata}";
	    d2tfigpreamble = "\tikzstyle{every state}= \
	    [draw=black!95,very thick,fill=white!20]";
	    node [style="state"];
	    edge [lblstyle="auto",topath="bend left"];
	    
		subgraph cluster_0 {
			label="Subgraph A";
			A -> B;
			B -> C;
			C -> D;
		}
	    
	    A [style="state"];
		APA [style="state"];
		A -> APA;
	    A -> B [label=2];
	    A -> D [label=7];
	    B -> A [label=1,style="-o"];
	    B -> B [label=3,topath="loop above"];
	    B -> C [label=4];
	    C -> F [label=5];
	    F -> B [label=8];
	    F -> D [label=7];
	    D -> E [label=2];
	    E -> A [label="1,6"];
	    F [style="state,accepting"];
	    B [style="transitionV, label=$\sin(x)$", label=""];
        }
    \end{dot2tex}
\end{tikzpicture}
\end{document}